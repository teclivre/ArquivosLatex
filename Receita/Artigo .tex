\documentclass[10pt,a4paper]{article}
\usepackage[utf8x]{inputenc}
\usepackage[brazil]{babel}
\usepackage{ucs}
\usepackage{amsmath}
\usepackage{amsfonts}
\usepackage{amssymb}
\usepackage{makeidx}
\usepackage{graphicx}
\usepackage{cooking,textcomp}
 
\pagestyle{recipe}
\begin{document}
 
\begin{recipe}{\textbf{Bolo de Cenoura}| (rende 8~Porçoes)}
\ingredient{1/2 xicara de cha de oleo} 
\ingredient{3 cenouras medias raladas}
\ingredient{4 ovos}
\ingredient{2 chicaras de cha de açucar}
Bata no liquidificador primeiro a cenoura com os ovos e o óleo, acrescente o açúcar e bata por uns 5 minutos.
\ingredient{2 1/2 xícaras (chá) de farinha de trigo}
Depois numa tigela ou na batedeira, coloque o restante dos ingredientes misturando tudo, menos o fermento.
\ingredient{1 colher (sopa) de fermento em pó}
Esse e misturado lentamente com uma colher.
 
Asse em forno pré-aquecido (180ºC) por 40 minutos
 
\subsubsection*{Cobertura}
 
\ingredient{1 colher (sopa) de manteiga}
\ingredient{3 colheres (sopa) de chocolate em po ou achocolatado}
\ingredient{1 xicara (cha) de açucar}
\ingredient{Para uma cobertura mais mole coloque 5 colheres de leite}
 
Misture todos os ingredientes, leve ao fogo, faça uma calda e coloque por cima do bolo
Se o seu liquidificador for bem potente, o bolo todo pode ser feito nele
 
\item [Tempo de Preparo] 1 hora
 
\end{recipe}
\end{document}