%%%%%%%%%%%%%%%%%%%%%%%%%%%%%%%%%%%%%%%%%%%%%%%%%%%%%%%%
% Exemplo 06-05: newtheorem: criando ambientes de teoremas
% ultima atualizacao: 05/02/2018 por Sadao Massago
% http:/www.dm.ufscar.br/~sadao
%----------------------------------------------------
% EXEMPLO não finalizado
% talvez utilizar o AMS

\documentclass[12pt,a4paper]{article}

% T1 não é aceito no PCTeX 4.0 ou anterior.
% Neste caso, comente
\usepackage[T1]{fontenc} % codificação da fonte em 8-bits
\usepackage[utf8]{inputenc} % acentuação direta
\usepackage[brazil]{babel} % em portugues brasileiro
\usepackage{amsthm}

% \usepackage{theorem} % para configurar o teorema
\usepackage{amsthm} % para configurar o teorema (do AMS)

% \theoremstyle{plain} % o padrao. Ja esta com este estilo

\newtheorem{thm}{Teorema}[section] % usar contador associado a secao
\newtheorem{cor}[thm]{Corolário} % usar mesmo contador do thm
\newtheorem{lem}[thm]{Lema} % usar mesmo contador do thm
\newtheorem{prop}[thm]{Proposição} % usar mesmo contador do thm
\newtheorem{axi}[thm]{Axioma} % usar mesmo contador do thm

%\theorembodyfont{\normalfont\upshape} % agora, nao eh mais fonte italico

\theoremstyle{definition}
\newtheorem{defn}[thm]{Definição} % usar mesmo contador do thm
\newtheorem{ex}[thm]{Exemplo} % usar mesmo contador do thm

\newtheorem{exerc}{Exercício}[section] % usar contador associado a secao

\theoremstyle{remark}
\newtheorem{obs}[thm]{Observação} % Observação
\newtheorem{TESTE}{Definicão}
% ambiente de demonstração
%\newenvironment{dem}[1][Demonstração]{\textbf{#1:}\ }  {\hfill\rule{1ex}{1ex}}
% amsthm ja tem ambiente definido chamado proof
%------------------------------------------------------------------------------
% regra de hifenização das palavras nao acentuadas:
% nao requer \usepackage[T1]{fontenc}
\hyphenation{li-vro tes-te cha-ve bi-blio-te-ca}
% regra de hifenização das palavras acentuadas:
% requer \usepackage[T1]{fontenc}
\hyphenation{co-men-t\'a-rio re-fe-r\^en-cia}

\begin{document}

\section{Teorema de Pitágoras}

\subsection{Triângulo Retângulo}
O ambiente similar a demonstração pode ser criado pelo \texttt{newtheorem}. Assim, ele será enumerada automaticamente e poderá controlar o estilo dele, usando o pacote \texttt{amsthm} (melhor do que o pacote \texttt{theorem}).

Os ambientes ``theorem like'' abaixo  são exemplos destes.

\begin{defn}[Triângulo Retângulo]
  Um triângulo $\Delta ABC$ é dito \emph{triângulo retângulo} quando um dos vêrtices 
  apresenta ângulo reto.
\end{defn}

\subsection{Resultados}
\begin{thm}[Pitagoras]
  Sejam $\Delta ABC$, um triângulo retângulo onde $a$ onde $a$ é hipotenusa. Então
  \[ a^2=b^2+c^2 \] 
\end{thm}

\begin{proof}
  ...
\end{proof}

\begin{cor}
 Seja $\Delta ABC$, o triângulo retângulo com hipotenusa $a$. Então $a > b+c$
\end{cor}

\begin{proof}
  ...
\end{proof}

\begin{obs}
Observação aqui...
\end{obs}

\subsection{Aplicações}

Colocar aplicações aqui.

\subsection{Exercícios}

\begin{exerc}
  Prove que o triângulo com lados $3,4$ e $5$ é retângulo.
\end{exerc}
\begin{exerc}
  Prove que a altura do triângulo equilátero com lado $l$ é $\frac{\sqrt{2}}{2}$.
\end{exerc}

\end{document} % final do documento
