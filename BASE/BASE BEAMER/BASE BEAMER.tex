\documentclass[aspectratio=169, 12pt]{beamer}
%\usepackage[top=3cm, botton=2cm, left=3cm, right=2cm]{geometry}
\usepackage[utf8]{inputenc}
\usetheme{Madrid}
%\usecolortheme{albatross}
%\usetheme[color= blue]{epyt}
%\usetheme[showheader, red, gray, graytitle, colorblocks, noframetitlerule]{Verona}
%\usepackage{metropolis}
%\usetheme{PhnomPenh}
\usepackage{graphicx}
\usepackage[brazilian]{babel}
\usepackage{fancyhdr}
\usepackage{amsmath, amsfonts, amssymb}
\usepackage{epigraph}
\title{Matamática Básica}
\author{Roberio Figueiredo}
\institute [TECLIVRE] {http://www.teclivre.com}
%\date{2016}
\logo{\includegraphics[scale=0.05]{logo}}
\title[\sc{Potência de 10}]{TECLIVRE}


\begin{document}
\transdissolve

\begin{frame}  \transboxin
\titlepage
\end{frame}

\begin{frame}  %\transboxin 
\transdissolve
\frametitle{Matemática Básica}
\transdissolve
\framesubtitle{Regras de Potências} \pause
\transdissolve
\begin{block}{Potências de \alert{10}} \pause
\transdissolve

Este tipo de potência, muito inportante em cálculos envolvendo grandezas, sejam elas muito grandes ou muito pequenas. \pause
\transdissolve

São muito utilizadas em estudos cientificos, como no caso da física e química e engenharia.
\end{block} \pause
\newpage
\end{frame}
\transdissolve


\frame{
\begin{columns}
\column{4cm}
\begin{block}{Expoênte positivo} \pause
\transdissolve
$ 10^0 =$ \pause \alert{$1$} \pause

$10^1 =$ \pause \alert{$10$} \pause

$10^2 =$ \pause \alert{$100$} \pause

$10^3 =$ \pause \alert{$1.000$} \pause

$10^4 =$ \pause \alert{$10.000$} \pause

$10^5 =$ \pause \alert{$100.000$} \pause

$10^6 =$ \pause \alert{$1.000.000$} \pause

$10^7 =$ \pause \alert{$10.000.000$} \pause

$10^n =$ \pause \alert{$10000...n$} \pause
\end{block}
\column{4cm}
\begin{block}{Expoênte negativo} \pause
\transdissolve
$ 10^0 =$ \pause \alert{$1$} \pause

$10^-1 =$ \pause \alert{$0,1$} \pause

$10^-2 =$ \pause \alert{$00,1$} \pause

$10^-3 =$ \pause \alert{$000,1$} \pause

$10^-4 =$ \pause \alert{$0000,1$} \pause

$10^-5 =$ \pause \alert{$00000,1$} \pause

$10^-6 =$ \pause \alert{$000000,1$} \pause

$10^-7 =$ \pause \alert{$0000000,1$} \pause

$10^-n =$ \pause \alert{$0000...,n$} \pause
\end{block}
\end{columns}


\begin{block}{NOTA}
\transdissolve
A quantidade de zeros do resultado, representa o expoênte da potência
\end{block}
}
\newpage

\transdissolve
\begin{frame}{Exemplo:}\pause

$5000$ \pause $\Longrightarrow$ $5 \cdot 10^3$ \pause

$0,005$ \pause $\Longrightarrow$ $5 \cdot 10^{-3}$ \pause $\Longrightarrow$ $\frac{1}{500}$ \pause

\end{frame}




\frame{
\begin{columns}
\column{3cm}
Contedo da coluna 1
\column{3cm}
Contedo da coluna 2
...
\end{columns}
}





\end{document}